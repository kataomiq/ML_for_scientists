
$\mathbb{N}$ denotes the set of natural numbers. We use
$\log{k}$ to denote $\log_{2}{k}$. For a set $S$, $|{S}|$
and $2^{{S}}$ denote its cardinality and power set,
respectively. The union of $A$ and $B$ is denoted as $A\cup B$. The union of two disjoint sets $A,B$ is denoted as $A \sqcup B$. The notation $[a,b]$ represents
$\{a,a+1\cdots b\}$ and $[b]$ represents $[1,b]$.
For every number $i$, we create distinct copies
$i_0,i_1,i_2\cdots$. The set $[w]_i$ represents
$\{1_i,2_i\cdots w_i\}$. Note that the sets
$[w],[w]_0,[w]_1\cdots$ are all different from each
other as these sets are pairwise disjoint. Given a family of
sets $\mathcal{S}=\{S_1,S_2\cdots S_t\}$ and a set $X$, we
define $\mathcal{S}+X= \{S_1\cup X,S_2\cup X\cdots S_t\cup X\}$.

An anti-chain is a subset $\mathcal{A}$ of a partially
ordered set $P$ such that any two distinct elements of
$\mathcal{A}$ are incomparable. An (order) ideal (also
called semi-ideal, down-set, or monotone decreasing subset)
of $P$ is a subset $I$ of $P$ such that if $t \in I$ and
$s \leq t$, then $s \in I$. Similarly, a dual order ideal
(also called up-set or monotone increasing subset) is a
subset $I$ of $P$ such that if $t \in I$ and $s \geq t$,
then $s \in I$ \cite{stanley}. When $P$ is finite, there is
a one-to-one correspondence between anti-chains of $P$ and
order ideals: the anti-chain $\mathcal{A}$ associated with
the order ideal $I$ is the set of maximal elements of $I$,
while $I = \{\, s \in P \mid s \leq t \text{ for some }
t \in \mathcal{A} \,\}$. Then the anti-chain $\mathcal{A}$
is said to generate the ideal $I=\mathbf{ID}(\mathcal{A})$.


Take the universe $U = [m] = \{1,\ldots,m\}$. Given any subset
$S$ of $U$, we can associate to $S$ a term
$T_S= \bigwedge_{i \in S} x_i$. Conversely, given a term
$T= \{ x_{i_1} \land x_{i_2} \land \ldots \land x_{i_t} \}$
of a monotone DNF formula, we can associate the subset
$S_T = \{i_1, i_2, \ldots, i_t\}$ to $T$. Also, to a family
$\mathcal{F} = \{ S_1, S_2, \ldots, S_t\}$ of subsets of $U$,
we can associate a monotone DNF formula
$f_{\mathcal{F}} = T(S_1) \lor T(S_2) \lor \ldots \lor T(S_t)$.
Conversely, to a monotone DNF formula
$f = T_1 \lor T_2 \lor \ldots \lor T_{\ell}$, we can associate
a family of subsets of $U$, namely
$\mathcal{F}_f = \{S_{T_1}, S_{T_2}, \ldots, S_{T_\ell} \}$.


Let $\mathcal{S},\mathcal{T}$ be family of sets such that $S\cap
T=\emptyset$ for all $S\in \mathcal{S}$ and $T\in \mathcal{T}$, where
  $$|\mathcal{S}|=\alpha(m) \And |\mathbf{ID}(\mathcal{S})|=m$$
  $$|\mathcal{T}|=\alpha(k+1) \And |\mathbf{ID}(\mathcal{T})|=k+1$$

  Observe that, by construction $\mathbf{ID}(\mathcal{S}) \cap
  \mathbf{ID}(\mathcal{T})=\{\emptyset\}$. Therefore,
  $$|\mathbf{ID}(\mathcal{S} \cup \mathcal{T})| = |\mathbf{ID}(\mathcal{S})|+
  |\mathbf{ID}(\mathcal{T})|-1=m+k$$

  It now follows that,
  $$\alpha(m+k) \leq |\mathcal{S} \cup \mathcal{T}|=|\mathcal{S}|+
  |\mathcal{T}| = \alpha(m)+\alpha(k+1) $$


\begin{itemize}
\item The strategy only associates elements in $\{1 \ldots \ell\}$ to elements in $\{1\ldots \ell+1\}$ and vice versa, so the unary predicate $N$ is respected;
\item The strategy associates endpoints to endpoints, so atomic formulas of the form $x = \textbf{1}$ or $x = \textbf{end}$ are respected;
\item Atoms of the form $E(x,\textbf{1})$ or $E(\textbf{end},x)$ are never satisfied in either model, so there is nothing to check about them;
\item Atoms of the form $E(\textbf{start},x)$ are satisfied (in $\M_\ell$ or $\M_{\ell+1}$) only if $x=2$, and this strategy maps $2$ in $\M_\ell$ to $2$ in $\M_{\ell+1}$ and vice versa;
\item Atoms of the form $E(x, \textbf{end})$ are satisfied in $\M_\ell$ only if $x = \ell-1$ and in $\M_{\ell+1}$ only if $x = \ell$, and this strategy maps $\ell-1$ in $\M_{\ell}$ to $\ell$ in $\M_{\ell+1}$;
\item In one move it is not possible to pick two indexes that are one the successor of each other (note that the elements of $A$ never occur in the relation $E$), so atoms of the form $Exy$ are trivially respected;
\item The $\mathfrak f$s are isomorphisms that keep $A$ fixed pointwise, so the relations $Q$ and $T$ are respected.
\end{itemize}
\item[Induction Case]: Consider now the game $EF_{n+1}(\M_{\ell}, \M_{\ell+1})$ where $\ell > 2^{n+2}$. We must consider a few different cases depending on the first move of Spoiler:
\begin{enumerate}
\item If Spoiler plays in $\M_\ell$ and picks an element $t \in \{1 \ldots 2^{n+1}+1\}$ or an element $b \in B_t$ for some $t \in \{1 \ldots 2^{n+1}+1\}$, answer with the same element in $\M_{\ell+1}$. Now, the submodels of these models that are associated to indexes no greater than $t$ are isomorphic to each other (their domains are both of the form $A \cup \{1 \ldots t\} \cup \bigcup \{B_1 \ldots B_t\}$), so for the rest of the game Duplicator can play along this isomorphism for these submodels. The parts of the two models that instead \emph{begin} from $t$ (included) are of the form $A \cup \{t \ldots \ell\} \cup \bigcup\{B_{t} \ldots B_{\ell}\}$ and $A \cup \{t \ldots \ell+1\} \cup \bigcup\{B_{t} \ldots B_{\ell+1}\}$ respectively; but these are isomorphic to $\M_{\ell - t+1}$ and $\M_{\ell-t+2}$ respectively (it is just a matter of gluing together the various isomorphisms from $A \cup B_{t}$ to $A \cup B_1$, from $A \cup B_{t+1}$ to $A \cup B_2$ et cetera, which we can do because they all agree over $A$). But $\ell - t + 1> 2^{n+2} - (2^{n+1}+1) +1 = 2^{n+1}$, so by induction hypothesis Duplicator can survive for $n$ turns by playing elements between these submodels.\footnote{Note that both sub-strategies associate $t$ to $t$ and $B_t$ to $B_t$, so there is no conflict between them insofar as they overlap; and that no atomic formula holds between an element that is picked according to one substrategy and one that doesn't, so if Duplicator can win both subgames she can win the game.}
\item If Spoiler plays in $\M_{\ell + 1}$ and picks an element $t \in \{1 \ldots 2^{n+1}+1\}$ or an element $b \in B_t$ for some $t \in \{1 \ldots 2^{n+1}+1\}$, answer with the same element in $\M_{\ell}$. By the same argument used above, Duplicator can then survive for $n$ more turns.
\item If Spoiler plays in $\M_\ell$ and picks an element $t \in \{2^{n+1}+2 \ldots \ell\}$, or an element $b \in B_t$ for some $t \in \{2^{n+1}+2 \ldots \ell\}$, answer with $t+1$ or $\mathfrak f_{t,t+1}(b)$ respectively.

This time, the submodels associated to indexes starting from $t$ and $t+1$ respectively are isomorphic to each other (and to $A \cup \{1 \ldots \ell - t+1\} \cup \bigcup\{B_1 \ldots B_{\ell-t+1}\}$); and the ones associated to indexes up to $t$ are equal to $\M_{t}$ and $\M_{t+1}$ respectively, and since $t > 2^{n+1}$ by induction hypothesis Duplicator can survive for $n$ turns between these two submodels.
\item If Spoiler plays in $\M_{\ell+1}$ and picks an element $t \in \{2^{n+1}+2 \ldots \ell+1\}$, or an element $b \in B_t$ for some $t \in \{2^{n+1}+2 \ldots \ell+1\}$, answer with $t-1$ or $\mathfrak f_{t-1,t}^{-1}(b)$ respectively. By the same argument used above, Duplicator can then survive for $n$ more turns.
\end{enumerate}



assertof{b} $\Rightarrow$ \assertof{a}: This can be proved by an argument similar to that
of the proof of \assertof{b} $\Rightarrow$ \assertof{a} of \Propof{p-ext-super-ext-2} below.

\assertof{a} $\Rightarrow$ \assertof{a$'$}:
follows from Lemmas \ref{p-ext-super-ext-0-0} and \ref{p-ext-super-ext-0-1}.\memo{Scan\_2024-11-23--10.44
 extendible - annotated p58 }
{\ifextended\extendedcolor\smallskip

 Assume that $\kappa$ is extendible, and suppose $\lambda>\kappa$. We want to show
 that there is $j$ as in \assertof{b} for this $\lambda$.

 Let $\lambda'>\lambda$ be a regular cardinal \st\ $V_\lambda\in\calH(\lambda')$, and let
 $\lambda''>\lambda$ be \st\ $\calH(\lambda'')=V_{\lambda''}$.

 By assumption there is $\Elembed{j''_0}{V_{\lambda''}}{V_{\mu''}}{\kappa}$ for
 some $\mu''$ with $j(\kappa)>\lambda''$ ($>\lambda$). Letting $j'_0:=j''_0\restr\calH(\lambda')$, we have
 $\Elembed{j'_0}{\calH(\lambda')}{\calH(j(\lambda'))}{\kappa}$.

 Let $N_0:=\bigcup j\imageof{\calH(\lambda')}$. Then we have
 $\Elembed{j'_0}{\calH(\lambda')}{N_0}{\kappa}$, and $j'_0$ is cofinal in $N_0$ by
 \Lemmaof{p-ext-super-ext-0-0}. By \Lemmaof{p-ext-super-ext-0-1}, $j'_0$ has a lifting
 $j\supseteq j'_0$ with $\Elembed{j}{\uniV}{M}{\kappa}$ for some transitive
 $M\subseteq\uniV$. Since $j(\lambda)=j''_0(\lambda)$, We have
 $j(V_\lambda)=V_{j(\lambda)}\in j(\ssetof{V_\lambda})\subseteq N_0\subseteq M$. Thus this
 $j$ is as desired.



\begin{Prop}\Label{p-ext-super-ext-2}
 For a cardinal $\kappa$ and $n\geq1$, the following are equivalent:\smallskip

 \wassert{a} For any $\lambda_0>\kappa$ there are $\lambda>\lambda_0$ with
 $V_\lambda\prec_{\Sigma_n}\uniV$, $j_0$, and $\mu$ \st\
 $\Elembed{j_0}{V_\lambda}{V_\mu}{\kappa}$, $j(\kappa)>\lambda$, and
 $V_\mu\prec_{\Sigma_n}\uniV$. \smallskip

 \wassert{b} For any $\lambda_0>\kappa$ there are $\lambda>\lambda_0$ with
 $V_\lambda\prec_{\Sigma_n}\uniV$, $j_0$, and $\mu$ \st\
 $\Elembed{j_0}{V_\lambda}{V_\mu}{\kappa}$, and
 $V_\mu\prec_{\Sigma_n}\uniV$ (without the condition ``$j(\kappa)>\lambda$''). \smallskip

 \wassert{a$'$} $\kappa$ is super-$C^{(n)}$-extendible.

 \wassert{b$'$} for any $\lambda_0>\kappa$ there are $\lambda\geq\lambda_0$ with
 $V_\lambda\prec_{\Sigma_n}\uniV$, and $j$, $M\subseteq\uniV$ \st\
 $\Elembed{j}{\uniV}{M}{\kappa}$, $V_{j(\lambda)}\in M$, and
 $V_{j(\lambda)}\prec_{\Sigma_n}\uniV$ (without the condition ``$j(\kappa)>\lambda$''). \qed
\end{Prop}



\begin{xitemize}
\xitem[x-ext-super-ext-1] for all sufficiently large
 $\lambda>\kappa$,
if \ixitemr[x-ext-super-ext-2] $V_\lambda\prec_{\Sigma_n}\uniV$, and $\mu$, $j$ are
 \st\ \ixitem[x-ext-super-ext-3] $\Elembed{j}{V_\lambda}{V_\mu}{\kappa}$ and
\ixitemr[x-ext-super-ext-4] $V_\mu\prec_{\Sigma_n}\uniV$, \\then $j(\kappa)<\gamma$.
\end{xitemize}
In the following, let $\gamma$ be the least such $\gamma$.

\begin{Claim}\Label{cl-ext-super-ext-0}
 $\gamma$ is a limit ordinal. For all sufficiently large $\lambda$ with \xitemof{x-ext-super-ext-2} and for
 all $\xi<\gamma$, there are $\mu$, $j$ with \xitemof{x-ext-super-ext-3},
 \xitemof{x-ext-super-ext-4} \st\ $j(\kappa)>\xi$.
\end{Claim}
\prfofClaim Suppose $\gamma$ is not a limit ordinal, say $\gamma=\xi+1$. Then there are
cofinally many $\lambda\in\On$
\st\ $V_\lambda\prec_{\Sigma_n}\uniV$ (actually $\lambda\in\Card$, see \Lemmaof{p-Lg}),
and there are $j$ and $\mu$ with
\xitemof{x-ext-super-ext-3}, \xitemof{x-ext-super-ext-4} and $j(\kappa)=\xi$.
By restricting of $j$'s as right above, it follows that, for all $\lambda>\xi$ with
$V_\lambda\prec_{\Sigma_n}\uniV$, there are $j$ and $\mu$ as above.

Let
$\lambda^*$ be a sufficiently large such $\lambda$ where ``sufficiently large'' is meant in
terms of \xitemof{x-ext-super-ext-1}. Let $j^*$ and $\mu^*$ be \st\
$\Elembed{j^*}{V_{\lambda^*}}{V_{\mu^*}}{\kappa}$, $j^*(\kappa)=\xi$, and $V_{\mu^*}\prec_{\Sigma_n}\uniV$.

Since $\lambda^*\leq\mu^*$, there is also $\Elembed{k}{V_{\mu^*}}{V_{\nu^*}}{\kappa}$ \st\
$V_{\nu^*}\prec_{\Sigma_n}\uniV$ and $k(\kappa)=\xi$. But then we have
$\Elembed{k\circ j^*}{V_{\lambda^*}}{V_{\nu^*}}{\kappa}$ and
$k\circ j^*(\kappa)=k(\xi)>k(\kappa)=\xi$. This is a contradiction to
\xitemof{x-ext-super-ext-1}.


Now, let $\lambda>\kappa$ be sufficiently large with $\lambda\geq\gamma+2$,
$V_\lambda\prec_{\Sigma_n}\uniV$, and $\Elembed{j}{V_\lambda}{V_\mu}{\kappa}$ with
$V_\mu\prec_{\Sigma_n}\uniV$. By \Claimabove, we have $j\imageof{\gamma}\subseteq\gamma$.

{\bf Case 1.} $\cf(\gamma)=\omega$. Then $j(\gamma)=\gamma$ and hence
$\Elembed{j\restr V_{\gamma+2}}{V_{\gamma+2}}{V_{\gamma+2}}{\kappa}$. This is a
contradiction to Kunen's proof (see e.g.\ Kanamori \cite{higher-inf}, Corollary 23.14).

{\bf Case 2.} $\cf(\gamma)>\omega$. then, letting $\kappa_0:=\kappa$,
$\kappa_{n+1}:=j(\kappa_n)$ for $n\in\omega$
and $\kappa_\omega:=\sup_{n\in\omega}\kappa_n$, we have $\kappa_\omega<\gamma$, and
$\Elembed{j\restr V_{\kappa_{\omega+2}}}{V_{\kappa_\omega+2}}{V_{\kappa_\omega+2}}{\kappa}$.
This is again a contradiction to Kunen's proof.


cardinal $\kappa$ is {\It$C^{(n)+}$-extendible} if for any $\lambda_0>\kappa$, there are
$\lambda\geq\lambda_0$ with $V_\lambda\prec_{\Sigma_n}\uniV$ and $j$, $M\subseteq V$ \st\
$\Elembed{j}{\uniV}{M}{\kappa}$ $j(\kappa)>\lambda$, $V_{j(\lambda)}\in M$,
and $V_{j(\kappa)}\prec_{\Sigma_n}V_{j(\lambda)}\prec_{\Sigma_n}\uniV$.
\fi}

The following notion is introduced by Benjamin Goodman \cite{goodman}.

A cardinal $\kappa$ is {\It supercompact for $C^{(n)}$} if, for any $\lambda>\kappa$ there is
$\Elembed{j}{\uniV}{M}{\kappa}$ \st\ $\fnsp{\lambda}{M}\subseteq M$ and
$C^{(n)}\cap\lambda=(C^{(n)})^M\cap\lambda$.




Suppose that $\kappa$ is super-$C^{(k+1)}$-extendible. Note
 that $\kappa$ is then super-$C^{(k)}$-extendible by \LemmaAof{p-ext-super-ext-4} and hence
 we have $\kappa\in C^{(k+2)}$ by the induction hypothesis. Let $\psi(x_0\ctenten)$ be a
 $\Pi_{(k+1)+2}$ formula. Say, $\psi(x_0\ctenten)=\exists x\varphi(x,x_0\ctenten)$ where
 $\varphi$ is a $\Pi_{k+2}$ formula. Then \ixitema[x-ext-super-ext-4-a] $\varphi$ is absolute over $V_\kappa$.

 In particular, if $V_\kappa\models\exists x\varphi(x,a_0\ctenten)$ for
 $a_0\ctenten\in V_\kappa$, then $\uniV\models\exists x\varphi(x,a_0\ctenten)$.

 Now suppose that $\uniV\models\exists x\varphi(x,a_0\ctenten)$ for
 $a_0\ctenten\in V_\kappa$. Let $\alpha>\kappa$ be
 \st\ \ixitema[x-ext-super-ext-4-0] $V_\alpha\prec_{\Sigma_{k+2}}\uniV$ and there is $c\in V_\alpha$ \st\
 $\uniV\models\varphi(c,a_0\ctenten)$. Since $\kappa$ is super-$C^{(k+1)}$-extendible,
 there are \ixitema[x-ext-super-ext-5] $\beta\in C^{(k+1)}$ and
 $\Elembed{j}{V_\alpha}{V_\beta}{\kappa}$ \st\ $j(\kappa)>\alpha$.

 Since $j(a_0)=a_0\ctenten$ and by \xitemof{x-ext-super-ext-5}, we have
 $V_\beta\models\varphi(c,j(a_0)\ctenten)$. It follows that
 $V_\beta\models\exists x\in V_{j(\kappa)}\ \varphi(x,j(a_0)\ctenten)$. By elementarity
 of $j$, $V_\alpha\models\exists x\in V_{\kappa}\ \varphi(x,a_0\ctenten)$.

 Let $c'\in V_\kappa$ be \st\ $V_\alpha\models \varphi(c',a_0\ctenten$). By
 \xitemof{x-ext-super-ext-4-0}, $\uniV\models \varphi(c',a_0\ctenten)$.
 Thus, by \xitemof{x-ext-super-ext-4-a}, it follows that
 $V_\kappa\models \varphi(c',a_0\ctenten)$, and hence
 $V_\kappa\models \psi(a_0\ctenten)$.



Now, let $\sigma<\mathfrak{c}$, and assume that we have already fixed sequences $(w_\tau)_{\tau<\sigma}$ and $(a_{\tau,\gamma})_{\gamma\leq\tau<\sigma}$ satisfying the following conditions:
 \begin{enumerate}[(i)]
\item\label{item::cond1} $w_\tau\in B_\tau$ for any $\tau<\sigma$,
\item $a_{\tau,\gamma}\in (\mathbb{R}\setminus C_\tau)$ for any $\gamma\leq\tau<\sigma$,
\item $a_{\tau,\gamma}\neq w_{\tau'}$ for any $\gamma\leq\tau<\sigma$ and $\tau'<\sigma$,
\item $a_{\tau,\gamma}\neq a_{\tau',\gamma'}$ for any $\gamma\leq\tau<\sigma$ and $\gamma'\leq \tau'<\sigma$ with $(\tau,\gamma)\neq(\tau',\gamma')$,
\item\label{item::cond5} $\{a_{\tau,\gamma}\mid \gamma\leq\tau<\sigma\}$ is algebraically independent over $K$.
 \end{enumerate}
 We now choose suitable $w_\sigma$ and $(a_{\sigma,\alpha})_{\alpha\leq\sigma}$.
 \begin{itemize}
\item Choice of $w_\sigma$: \\
The set $B_\sigma\in \mathcal{B}_0$ has positive measure $\mu(B_\sigma)>0$, and thus we obtain $|B_\sigma|=\mathfrak{c}$ by applying Kechris~\cite[Theorem~13.6]{Kechris}. On the other hand, we compute
$$|D_{\sigma,0}|=|\{a_{\tau,\gamma}\mid \gamma\leq\tau<\sigma\}|\leq|\sigma|^2<\mathfrak{c}.$$
Therefore, we can choose an element $w_\sigma$ from the set $B_\sigma\setminus D_{\sigma,0}$.
    
\item Choice of $(a_{\sigma,\alpha})_{\alpha\leq\sigma}$: \\
Let $\alpha\leq\sigma$, and assume that we have already fixed a sequence $(a_{\sigma,\gamma})_{\gamma<\alpha}$ in an appropriate way (i.e.\ in accordance with suitable extensions of conditions \eqref{item::cond1}--\eqref{item::cond5}). We compute
$$|D_{\sigma,\alpha}|=|\{a_{\tau,\gamma}\mid \gamma\leq\tau<\sigma\}|+|\{a_{\sigma,\gamma}\mid \gamma<\alpha\}|\leq |\sigma|^2+|\alpha|<\mathfrak{c}.$$
Since the transcendence degree of $\mathbb{R}$ over $K$ is $\mathfrak{c}$, there exists a set $T_{\sigma,\alpha}\subseteq\mathbb{R}$ with $|T_{\sigma,\alpha}|=\mathfrak{c}$ such that $D_{\sigma,\alpha}\mathbin{\dot{\cup}}T_{\sigma,\alpha}$ is algebraically independent over $K$. The complement of the set $C_\sigma\in\mathcal{B}_1$ has positive measure $\mu(\mathbb{R}\setminus C_\sigma)>0$. Thus, by Fact~\ref{fact::Steinhaus} the Minkowski difference $(\mathbb{R}\setminus C_\sigma)-(\mathbb{R}\setminus C_\sigma)$ is a neighborhood of $0$. Since multiplication with non-zero elements of $K$ preserves algebraic independence over $K$, we can without loss of generality assume that $T_{\sigma,\alpha}$ is a subset of $(\mathbb{R}\setminus C_\sigma)-(\mathbb{R}\setminus C_\sigma)$. Hence, any $t\in T_{\sigma,\alpha}$ can be written as $t=u-v$ for some $u,v\in (\mathbb{R}\setminus C_\sigma)$. Set $U_t=\{x\in \{u,v\}\mid D_{\sigma,\alpha}\mathbin{\dot{\cup}} \{x\}\text{ is algebraically independent over } K\}$, and note that $U_t$ is non-empty, as $t=u-v$ and $D_{\sigma,\alpha}\mathbin{\dot{\cup}} \{t\}$ is algebraically independent over $K$. Now, consider the set $$U_{\sigma,\alpha}= \bigcup\limits_{t\in T_{\sigma,\alpha}}U_t.$$ Then the union $(D_{\sigma,\alpha}\mathbin{\dot{\cup}} U_{\sigma,\alpha})\cup\{t\}$ is algebraically dependent over $K$ for any $t\in T_{\sigma,\alpha}$, i.e.\
$T_{\sigma,\alpha}$ is contained in the relative algebraic closure of the field $K(D_{\sigma,\alpha}\mathbin{\dot{\cup}} U_{\sigma,\alpha})\subseteq \mathbb{R}$. Since $|D_{\sigma,\alpha}|<|T_{\sigma,\alpha}|=\mathfrak{c}$, this yields
$|U_{\sigma,\alpha}|=\mathfrak{c}$. Moreover, we have $|\{w_\tau\mid \tau\leq\sigma\}|<\mathfrak{c}$. Therefore, we can choose an element $a_{\sigma,\alpha}$ from the set $U_{\sigma,\alpha}\setminus\{w_\tau\mid \tau\leq\sigma\}$.
 \end{itemize}

 For $\alpha<\mathfrak{c}$ we now define
 $$A_\alpha:=\{a_{\sigma,\alpha}\mid \alpha\leq\sigma<\mathfrak{c}\},$$
 and we verify the conditions \eqref{item::inner-zero}, \eqref{item::complement-inner-zero} and \eqref{item::algebraic-independence}:
 \begin{enumerate}[(I)]
\item Let $\alpha<\mathfrak{c}$. To derive $\mu_*(A_\alpha)=0$, it suffices to verify $w_\sigma\in B_\sigma\setminus A_\alpha$ for any $\sigma<\mathfrak{c}$. Indeed, this implies $B\not\subseteq A_\alpha$ for any set $B\in\mathcal{B}_0$. Thus, let $\sigma<\mathfrak{c}$. We have $w_\sigma\in B_\sigma$ by our choice of $w_\sigma$. To verify $w_\sigma\notin A_\alpha$ we have to show that $w_\sigma\neq a_{\tau,\alpha}$ for any $\alpha\leq\tau<\mathfrak{c}$. Thus, let $\alpha\leq\tau<\mathfrak{c}$. If $\tau<\sigma$, then our choice of $w_\sigma$ ensures that $w_\sigma\neq a_{\tau,\alpha}$, as $w_\sigma\notin D_{\sigma,0}$ but $a_{\tau,\alpha}\in D_{\sigma,0}$. If $\sigma\leq\tau$, then our choice of $a_{\tau,\alpha}$ ensures that $w_\sigma\neq a_{\tau,\alpha}$, as
$a_{\tau,\alpha}\notin\{w_{\tau'}\mid \tau'\leq \tau\}$.
    
\item Let $\alpha\leq\sigma<\mathfrak{c}$. Then $a_{\sigma,\alpha}\in A_\alpha\setminus C_\sigma$, and hence $A_\alpha\not\subseteq C_\sigma$. As a consequence, we obtain
\begin{align*}
  &\,\{C\in\mathcal{B}_1\mid A_\alpha\subseteq C\}\\
  =&\,\{C_\sigma\mid \sigma<\mathfrak{c}, A_\alpha\subseteq C_\sigma\} \\
  \subseteq&\,\{C_\sigma\mid \sigma<\alpha\},
\end{align*}
which implies $|\{C\in\mathcal{B}_1\mid A_\alpha\subseteq C\}|\leq|\alpha|<\mathfrak{c}$. Applying Lemma~\ref{lemma::complement-family-cardinality} therefore yields $\mu_*(\mathbb{R}\setminus A_\alpha)=0$.
    
\item Our choice of the sequence ensures that the set $D_{\sigma,\alpha}\cup\{a_{\sigma,\alpha}\}$ is algebraically independent over $K$ for any $\alpha\leq\sigma<\mathfrak{c}$, where $a_{\sigma,\alpha}$ is distinct from all elements of $D_{\sigma,\alpha}$. Thus, the members of the sequence $(A_\alpha)_{\alpha<{\mathfrak{c}}}$ are pairwise disjoint, and we obtain the algebraic independence of the union $\dot{\bigcup\limits_{\alpha<\mathfrak{c}}}\,A_\alpha$
over $K$. \qedhere
 \end{enumerate}


\begin{fact}[{{R.~Robinson~\cite[\S\,5]{R-Robinson}}}]\label{fact::robinson}
    Let $F$ be a real field that admits an archimedean ordering and let $t$ be transcendental over $F$. Then $\mathbb{Z}$ is $\emptyset$--definable in $F(t)$.
\end{fact}

\begin{corollary}\label{corollary:trans-IP}
    Let $F$ be a real field that admits an archimedean ordering and let $t$ be transcendental over $F$. Then $F(t)$ has the independence property.
\end{corollary}

Note that due to Hölder's Theorem\footnote{Applied to ordered fields, this theorem states that for any archimedean ordered field $(F,<_F)$, there is a \emph{unique} order-preserving $\mathcal{L}_{\mathrm{r}}$--embedding from $(F,<_F)$ into $(\mathbb{R},<)$. See Hölder~\cite[Erster Theil]{Hoelder}, Engler and Prestel~\cite[Proposition~2.1.1]{Engler-Prestel}.}, real fields that admit at least one archimedean ordering are precisely the subfields of $\mathbb{R}$ (up to $\mathcal{L}_{\mathrm{r}}$--i\-so\-mor\-phism).

\section{Building a Wild Ordered Subfield of the Reals}\label{sec::main}

In this section, we assume some familiarity with Borel $\sigma$--algebras in general topological spaces and refer the reader to Bogachev~\cite[\S\,6.2]{Bogachev2} for further details.

We now turn to the construction of a subfield $K$ of $\mathbb{R}$ that has the independence property and defines a non-Borel set $D\subseteq K$. In this context, given a subset $X\subseteq \mathbb{R}$, there are two natural candidates for Borel $\sigma$--algebras on $X$:
\begin{itemize}
    \item We can endow $X$ with the \textbf{trace $\sigma$--algebra}
    $$\mathcal{B}(X):=\{B\cap X\mid B\in\mathcal{B}(\mathbb{R})\}$$
    induced by $\mathcal{B}(\mathbb{R})$. Denoting by $\tau_\mathbb{R}$ the order topology\footnote{Recall that the order topology on $\mathbb{R}$ coincides with the Euclidean topology, and thus it generates $\mathcal{B}(\mathbb{R})$.} on $\mathbb{R}$, it follows from \cite[Lemma~6.2.4]{Bogachev2} that $\mathcal{B}(X)$ is precisely the Borel $\sigma$--algebra generated by the subspace topology $\{U\cap X\mid U\in\tau_\mathbb{R}\}$ on $X$ induced by~$\tau_\mathbb{R}$.
   
    \item The subset $X$ of $\mathbb{R}$ inherits the linear ordering of $\mathbb{R}$. Thus, $X$ is naturally endowed with the order topology $\tau_X$, and we can consider the Borel $\sigma$--algebra $\mathcal{B}(\tau_X)$ generated by $\tau_X$.
\end{itemize}

The following result shows that the two Borel $\sigma$--algebras considered above coincide.

\begin{lemma}\label{lemma::Borel-identities}
    For any subset $X\subseteq\mathbb{R}$, we have $\mathcal{B}(X)=\mathcal{B}(\tau_X)$.
\end{lemma}
\begin{proof}
    The inclusion $\mathcal{B}(\tau_X)\subseteq\mathcal{B}(X)$ follows from the fact that the generators of the $\sigma$--algebras satisfy $\tau_X\subseteq \{U\cap X\mid U\in\tau_\mathbb{R}\}$. For the proof of the other inclusion, we consider the set $$\Sigma:=\{B\in\mathcal{B}(\mathbb{R})\mid (B\cap X)\in\mathcal{B}(\tau_X)\}.$$
    Note that $\Sigma$ is a $\sigma$--algebra on $\mathbb{R}$. If we prove the inclusion $\mathcal{B}(\mathbb{R})\subseteq\Sigma$, then the claim of the lemma follows. Hence, it suffices to verify $\mathbb{R}_{\leq a}\in \Sigma$ for any $a\in\mathbb{R}$ (cf.\ Bogachev~\cite[Lemma~1.2.11]{Bogachev1}). Given $a\in\mathbb{R}$, set
    $b=\sup (\mathbb{R}_{\leq a}\cap X),$
    where the supremum is taken in $\mathbb{R}$. If $b\in X$, then $(\mathbb{R}_{\leq a}\cap X)=X_{\leq b}\in\mathcal{B}(\tau_K)$. Otherwise, for any $n\in\mathbb{N}$ there exists $b_n\in(\mathbb{R}_{\leq a}\cap X)$ such that $b-\frac{1}{n}<b_n< b\leq a$. Hence,
    $$\mathbb{R}_{\leq a}\cap X=\bigcup\limits_{n\in\mathbb{N}} X_{\leq b_n}\in\mathcal{B}(\tau_{X}).$$
    This yields $\mathbb{R}_{\leq a}\in\Sigma$, as required.
\end{proof}








